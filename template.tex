% !TEX TS-program = xelatex
% !TEX encoding = UTF-8 Unicode

% \documentclass[AutoFakeBold]{LZUThesis}
\documentclass[AutoFakeBold]{LZUThesis}
% \setCJKfamilyfont{Noto}{Noto Sans CJK SC}

\begin{document}
%=====%
%
%封皮页填写内容
%
%=====%

% 标题样式 使用 \title{{}}; 使用时必须保证至少两个外侧括号
%  如: 短标题 \title{{第一行}},  
% 	      长标题 \title{{第一行}{第二行}}
%             超长标题\tiitle{{第一行}{...}{第N行}}

\title{{基于Deep Koopman算子网络的}{非线性系统强化学习研究}}



% 标题样式 使用 \entitle{{}}; 使用时必须保证至少两个外侧括号
%  如: 短标题 \entitle{{First row}},  
% 	      长标题 \entitle{{First row}{ Second row}}
%             超长标题\entitle{{First row}{...}{ Next N row}}
% 注意:  英文标题多行时 需要在开头加个空格 防止摘要标题处英语单词粘连。
\entitle{{Deep Koopman Network Based}{ Reinforcement Learning of Nonlinear System}}

% \author{{\CJKfamily{Noto} \zihao{3} 许忞欢}}
\author{{许忞欢}}
\major{电子信息科学与工程}
\advisor{赵东东}
\college{信息科学与工程学院}
\grade{2020级}



\maketitle

%==============================%
% ↓ ↓ ↓ 诚信说明页 授权说明书
%==============================%

% 1. 可以调整签字的宽度,现在是40
% 2. 去掉raisebox的相关注释(注意上下大括号对应),可以改变-5那个数字调整签名和横线的上下位置

% 你的签名,signature.pdf 改为你的签名文件名,
\mysignature{
    % \raisebox{-5pt}{
    \includegraphics[width=40pt]{signature.pdf}
    % }
}
% 你手写的日期,signature.pdf 改为你的手写的日期文件名
\mytime{
    % \raisebox{-5pt}{
    \includegraphics[width=40pt]{signature.pdf}
    % }
}
% 老师的手写签名,signature.pdf 改为老师的手写签名文件名
\supervisorsignature{
    % \raisebox{-5pt}{
    \includegraphics[width=40pt]{signature.pdf}
    % }
}
% 老师手写的时间,signature.pdf 改为老师的手写的日期文件名
\teachertime{
    % \raisebox{-5pSt}{
    \includegraphics[width=40pt]{signature.pdf}
    % }
}
% 老师手写的成绩
\recommendedgrade{
    % \raisebox{-5pt}{
    \includegraphics[width=40pt]{signature.pdf}
    % }
}

\makestatement

%==============================%
% ↑ ↑ ↑ 诚信说明页 授权说明书
%==============================%


%=====%
%论文(设计)成绩:注意2007的模板要求,成绩页在最后,2021要求成绩页在摘要前面
%=====%

% 下面这些注释掉可以去掉成绩、评语什么的
\supervisorcomment{好好好}


\committeecomment{优秀}

\finalgrade{100}
% 上面这些注释掉可以去掉成绩、评语什么的


\frontmatter



%中文摘要
\ZhAbstract{我的摘要}{Koopman算子理论,深度神经网络,强化学习}


%英文摘要
\EnAbstract{My Abstract}{Koopman Operator Theory, Deep Neural Network, Reinforcement Learning}

%生成目录
\tableofcontents
% 下面这个包含图表目录
% \customcontent


% 部分同学需要专业术语注释表,* 表示不加入目录
% \chapter*{专业术语注释表}
% \begin{longtable}{lll}
%   \caption*{缩略词说明}\\
%   SS & Spread Spectrum & 扩展频谱 \\
%   PAPR & Peak to Average Power Ratio & 峰均比\\
%   DCSK & Differential Chaos Shift Keying &差分混移位键控\\
%   dasd & fdhfudw eqwrqw fasfasfs fewev wqfwefew &\tabincell{l}{太长了\\换行一下}\\
% \end{longtable}


%文章主体
\mainmatter

\chapter{\texorpdfstring{绪 \quad 论}{绪论}}

这是我的绪论\cite{tenne1992polyhedral}

\chapter{背景知识}
在本章中,首先讨论一下有关的背景理论与算法。介绍一下Koopman算子理论(Koopman Operator Theory),并讨论Koopman算子对于重塑强化学习(Reinforcement Learning)中使用的马尔可夫决策过程(Markov Decision Process)的重要作用。同时,对于Koopman算子理论与深度神经网络(Deep Neural Network)之间的关联。

\section{Koopman算子理论}
系统的强非线性是数据驱动建模和控制领域的核心问题之一,包括现代强化学习框架

%论文后部
\backmatter


%=======%
%引入参考文献文件
%=======%
\bibdatabase{bib/database}%bib文件名称 仅修改bib/ 后部分
\printbib
% \nocite{*} %显示数据库中有的,但是正文没有引用的文献

\Appendix

这里是我的附录
这里是我的附录

这里是我的附录

\Thanks

这里是致谢页

(我是谁?兰朵儿开发者:余航,致谢我,查重时一定会重复的,哈哈,开个玩笑,本科生论文不在查重范围,而且“毕业论文(设计)检测内容主要为毕业论文(设计)的主体部分”)。


\Grade %这一句才是成绩页,上面是填写


\end{document}
